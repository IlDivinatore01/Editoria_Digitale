\subsection{Triathlon}

\begin{enumerate}

\item Essence of Triathlon
    \begin{itemize}
    \item Paralympic triathlon is the ultimate test of endurance and versatility, combining swimming, cycling, and running into a single, grueling event. 
    \item Athletes with various impairments showcase their strength, stamina, and determination as they navigate the challenging course, transitioning seamlessly between the three disciplines. 
    \item The sport celebrates the indomitable spirit of Paralympic athletes and their ability to push beyond limits.
    \end{itemize}

\item Rules, Equipment, and Competition
    \begin{itemize}
    \item Paralympic triathlon follows the core principles of triathlon, with adaptations for athletes with impairments. 
    \item The distances for each discipline vary depending on the classification, and athletes can use adaptive equipment such as handcycles, racing wheelchairs, or prosthetic limbs. 
    \item The competition format typically involves a continuous race, with athletes transitioning between the swim, bike, and run segments. 
    \end{itemize}

\item Categories and Classifications
    \begin{itemize}
    \item Paralympic triathlon features a classification system that groups athletes based on their impairment and functional abilities. 
    \item The system considers factors such as muscle power, range of movement, coordination, and balance. 
    \item Athletes are categorized into different classes, such as PTWC for wheelchair users, PTS for athletes with severe impairments who use prosthetics or assistive devices, and PTVI for visually impaired athletes who compete with a guide. 
    \item This classification ensures fair competition and allows athletes with diverse abilities to showcase their multi-sport skills.
    \end{itemize}

\end{enumerate}