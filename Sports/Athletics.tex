\subsection{Athletics}

\begin{enumerate}

\item Essence of Athletics
    \begin{itemize}
    \item Paralympic athletics encompasses a wide range of track, field, and road events, pushing athletes with various impairments to the limits of their physical capabilities. 
    \item It's a showcase of speed, power, endurance, and technical skill, with athletes competing in events like sprints, long jump, discus throw, and marathons. 
    \item Paralympic athletics demonstrates the extraordinary feats that can be achieved through dedication, training, and the unwavering pursuit of personal bests.
    \end{itemize}

\item Rules, Equipment, and Competition
    \begin{itemize}
    \item Paralympic athletics adheres to the fundamental rules of athletics, with adaptations made to accommodate different impairments. 
    \item Athletes can utilize adaptive equipment such as racing wheelchairs, running blades, and throwing frames to maximize their performance. 
    \item Competitions typically involve heats, semifinals, and finals, creating a thrilling atmosphere as athletes strive for podium finishes and record-breaking performances.
    \end{itemize}

\item Categories and Classifications
    \begin{itemize}
    \item Paralympic athletics employs a complex classification system that takes into account various impairments, including visual, physical, and intellectual disabilities. 
    \item Athletes are assessed based on their functional abilities, ensuring fair competition among those with similar levels of impairment. 
    \item This system uses a combination of letters and numbers to categorize athletes, such as T/F classes for track and field events, allowing spectators to understand the specific classifications and appreciate the unique challenges and achievements of each athlete.
    \end{itemize}
\end{enumerate}