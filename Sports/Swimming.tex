\subsection{Swimming}

\begin{enumerate}

\item Essence of Swimming
    \begin{itemize}
    \item Paralympic swimming is a showcase of grace, power, and determination, as athletes with various impairments compete in a variety of swimming strokes and distances. 
    \item It demands exceptional technique, strength, and endurance, as swimmers navigate the water with speed and precision. 
    \item The sport celebrates the adaptability and resilience of athletes with disabilities, demonstrating their ability to overcome challenges and achieve excellence in the aquatic environment.
    \end{itemize}

\item Rules, Equipment, and Competition
    \begin{itemize}
    \item Paralympic swimming adheres to the fundamental rules of swimming, with modifications made to accommodate athletes with impairments. 
    \item Swimmers compete in different strokes (freestyle, backstroke, breaststroke, butterfly), distances, and events (individual, relay). 
    \item The competition format typically involves heats, semifinals, and finals. 
    \item Adaptive equipment such as starting platforms, lane ropes, and tapping devices for visually impaired swimmers are used to ensure safe and fair competition.
    \end{itemize}

\item Categories and Classifications
    \begin{itemize}
    \item Paralympic swimming utilizes a complex classification system that considers the athlete's physical, visual, or intellectual impairment and its impact on their swimming ability. 
    \item Athletes are categorized into different classes based on their functional abilities, with each class representing a specific range of impairment. 
    \item This system ensures fair competition and allows athletes with diverse abilities to showcase their swimming skills and achieve their personal bests.
    \end{itemize}

\end{enumerate}