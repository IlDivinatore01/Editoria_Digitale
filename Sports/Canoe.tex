\subsection{Canoe (Para Canoe)}

\begin{enumerate}

\item Essence of Canoe
    \begin{itemize}
    \item Paralympic canoe, also known as para-canoe, is a thrilling water sport that showcases the strength, endurance, and paddling technique of athletes with physical impairments. 
    \item It involves paddling a canoe or kayak over a set distance, demanding a combination of upper body power, core stability, and precise boat control. 
    \item The sport offers both sprint and long-distance events, providing opportunities for athletes to excel in different paddling disciplines.
    \end{itemize}

\item Rules, Equipment, and Competition
    \begin{itemize}
    \item Paralympic canoe follows similar rules to Olympic canoe, with modifications made to accommodate different impairments. 
    \item Athletes compete in various boat classes, including kayaks (K) and va'a (V), and the competition format typically involves heats and finals. 
    \item The sport utilizes adaptive equipment, such as specialized seats and paddles, to ensure athletes with various impairments can participate safely and effectively.
    \end{itemize}

\item Categories and Classifications
    \begin{itemize}
    \item Paralympic canoe employs a classification system based on the athlete's physical impairment, ensuring fair competition among those with similar functional abilities. 
    \item The system uses a combination of letters and numbers to categorize athletes, such as KL for kayak lower limb impairments and VL for va'a lower limb impairments. 
    \item The classifications consider factors like muscle power, trunk function, and balance, allowing athletes to compete against others with comparable levels of impairment and demonstrate their paddling prowess.
    \end{itemize}

\end{enumerate}