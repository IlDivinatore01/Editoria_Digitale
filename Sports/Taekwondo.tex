\subsection{Taekwondo}

\begin{enumerate}
\item Essence of Taekwondo
    \begin{itemize}
    \item Paralympic taekwondo is a dynamic and exciting martial art that showcases the power, precision, and agility of athletes with physical impairments affecting their lower limbs. 
    \item Standing on one leg, athletes deliver powerful kicks and punches to score points against their opponents. 
    \item The sport demands exceptional balance, coordination, and tactical awareness, demonstrating the remarkable adaptability and fighting spirit of athletes with disabilities.
    \end{itemize}

\item Rules, Equipment, and Competition
    \begin{itemize}
    \item Paralympic taekwondo follows the basic rules of taekwondo, with modifications to accommodate athletes with impairments. 
    \item Matches are contested in a standing position, with athletes wearing protective gear and scoring points by landing kicks and punches on designated target areas. 
    \item The competition format involves different weight categories and knockout rounds, culminating in thrilling finals where athletes battle for gold.
    \end{itemize}

\item Categories and Classifications
    \begin{itemize}
    \item Paralympic taekwondo currently includes only one sport class, K44, for athletes with impairments in their upper limbs. 
    \item However, the sport is continuously evolving, and future editions of the Games may include additional classifications to accommodate athletes with other impairments. 
    \item The current classification system ensures fair competition among athletes with similar functional abilities and allows them to showcase their taekwondo skills.
    \end{itemize}

\end{enumerate}