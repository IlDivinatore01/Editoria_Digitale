\subsection{Equestrian (Para-Dressage)}

\begin{enumerate}

\item Essence of Equestrian
    \begin{itemize}
    \item Paralympic equestrian, also known as para-dressage, is a graceful and elegant sport that highlights the harmony between horse and rider. 
    \item Athletes with physical impairments showcase their skill, balance, and communication with their equine partners as they perform a series of intricate movements and patterns. 
    \item This sport demonstrates the unique bond between humans and animals and the transformative power of therapeutic riding.
    \end{itemize}

\item Rules, Equipment, and Competition
    \begin{itemize}
    \item Paralympic equestrian is governed by the rules of the International Equestrian Federation (FEI), with adaptations for athletes with impairments. 
    \item Competitions involve individual and team events, with riders performing dressage tests at different levels of difficulty. 
    \item Horses are carefully selected and trained, and adaptive equipment such as mounting blocks and specialized saddles may be used to assist riders with specific needs.
    \end{itemize}

\item Categories and Classifications
    \begin{itemize}
    \item Paralympic equestrian uses a classification system based on the rider's physical impairment and functional abilities. 
    \item There are five grades, ranging from Grade I for riders with the most severe impairments to Grade V for those with minimal impairments. 
    \item The classification process assesses factors like muscle strength, coordination, balance, and vision, ensuring that riders compete against others with similar functional abilities.
    \end{itemize}

\end{enumerate}