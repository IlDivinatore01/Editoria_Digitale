\subsection{Rowing}

\begin{enumerate}

\item Essence of Rowing
    \begin{itemize}
    \item Paralympic rowing is a test of strength, endurance, and teamwork, featuring athletes with physical impairments competing in boats on calm water courses. 
    \item It demands synchronized movements, powerful strokes, and unwavering focus, as rowers propel their boats towards the finish line. 
    \item The sport showcases the beauty of teamwork and the remarkable athleticism of individuals with disabilities.
    \end{itemize}

\item Rules, Equipment, and Competition
    \begin{itemize}
    \item Paralympic rowing follows similar rules to Olympic rowing, with adaptations for athletes with impairments. 
    \item Rowers compete in different boat classes, including single sculls, double sculls, mixed coxed fours, and mixed double sculls. 
    \item The competition format involves heats and finals, with rowers striving to achieve the fastest times. 
    \item Adaptive equipment such as specialized seats and oars are used to enable athletes with various impairments to participate and compete effectively.
    \end{itemize}

\item Categories and Classifications
    \begin{itemize}
    \item Paralympic rowing utilizes a classification system based on the athlete's functional abilities. 
    \item There are three main categories: 
        \begin{itemize}
        \item PR1 for athletes with limited trunk and arm movement
        \item PR2 for those with limited trunk movement but good arm and shoulder function
        \item PR3 for those with trunk and arm movement but limited leg or lower body function
        \end{itemize}
    \item This system ensures fair competition and allows athletes with different impairments to showcase their rowing skills.
    \end{itemize}

\end{enumerate}