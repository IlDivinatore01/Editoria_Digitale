\subsection{Cycling}

\begin{enumerate}

\item Essence of Cycling
    \begin{itemize}
    \item Paralympic cycling is a thrilling display of speed, endurance, and technical skill, featuring athletes with various impairments competing on bicycles, tricycles, handcycles, or tandems.
    \item The sport encompasses a range of disciplines, including road races, time trials, track cycling, and mountain biking, offering opportunities for athletes to excel in different cycling styles and terrains. 
    \end{itemize}

\item Rules, Equipment, and Competition
    \begin{itemize}
    \item Paralympic cycling follows similar rules to Olympic cycling, with adaptations made to accommodate different impairments. 
    \item Athletes compete in different classes based on their impairment and the type of bike they use. Competitions can be individual or team events, and they involve various formats such as time trials, road races, and pursuit races. 
    \item Adaptive equipment plays a crucial role, with specialized bikes, handcycles, and tandems enabling athletes to participate and compete at their best.
    \end{itemize}

\item Categories and Classifications
    \begin{itemize}
    \item Paralympic cycling uses a comprehensive classification system based on the athlete's impairment and functional abilities. 
    \item The system considers factors such as muscle power, range of movement, coordination, and vision. 
    \item Athletes are categorized into different classes, such as C for cyclists with physical impairments and B for visually impaired cyclists who compete on tandems with a sighted pilot. 
    \item This classification ensures fair competition and allows athletes to compete against others with comparable levels of impairment.
    \end{itemize}

\end{enumerate}