\subsection{Para Powerlifting}

\begin{enumerate}

\item Essence of Para Powerlifting
    \begin{itemize}
    \item Para powerlifting is a demonstration of raw strength and determination, where athletes with physical impairments compete to lift the heaviest possible weight in a bench press. 
    \item It requires immense power, technique, and mental fortitude, as athletes push their bodies to the limit in pursuit of lifting extraordinary weights. 
    \item The sport celebrates the strength and resilience of athletes with disabilities, showcasing their ability to overcome challenges and achieve remarkable feats.
    \end{itemize}

\item Rules, Equipment, and Competition
    \begin{itemize}
    \item Para powerlifting follows the core rules of powerlifting, with adaptations for athletes with impairments. 
    \item Athletes lie on a bench and attempt to lift a barbell loaded with weights, using only their upper body strength. 
    \item The competition involves three attempts, and the highest successful lift is counted. 
    \item The sport utilizes specialized equipment, such as bench press stations and assistive devices, to ensure safe and fair competition for athletes with different impairments.
    \end{itemize}

\item Categories and Classifications
    \begin{itemize}
    \item Para powerlifting employs a classification system based on the athlete's body weight and the impact of their impairment on their lifting ability. 
    \item Athletes are grouped into different weight categories, and within each category, they are further classified based on the functional impact of their impairment. 
    \item This ensures fair competition among athletes with similar levels of impairment and allows them to showcase their strength and technique on a level playing field.
    \end{itemize}

\end{enumerate}