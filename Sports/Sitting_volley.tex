\subsection{Sitting Volleyball}

\begin{enumerate}

\item Essence of Sitting Volleyball
    \begin{itemize}
    \item Sitting volleyball is a fast-paced and exciting team sport adapted for athletes with physical impairments affecting their lower limbs. 
    \item Players compete on a smaller court, sitting on the floor, and use their upper body strength, agility, and teamwork to volley the ball over the net and score points. 
    \item The sport showcases the incredible athleticism and competitive spirit of athletes with disabilities, demonstrating their ability to adapt and excel in a challenging and dynamic environment.
    \end{itemize}

\item Rules, Equipment, and Competition
    \begin{itemize}
    \item Sitting volleyball follows the basic rules of volleyball, with modifications to accommodate athletes with impairments. 
    \item The court is smaller, the net is lower, and players must maintain contact with the floor when playing the ball. 
    \item Teams consist of six players, and the game is played in sets, with the first team to reach a certain number of points winning the set. 
    \item The sport utilizes a standard volleyball and requires players to have at least one buttock in contact with the floor at all times.
    \end{itemize}

\item Categories and Classifications
    \begin{itemize}
    \item Sitting volleyball has a classification system that groups athletes based on the impact of their impairment on their playing ability. 
    \item There are two main classifications:  
        \begin{itemize}
        \item VS1 for athletes with minimal impairment
        \item VS2 for athletes with more significant impairments
        \end{itemize}
    \item This system ensures fair competition and allows athletes with different functional abilities to participate and compete against others with comparable levels of impairment.
    \end{itemize}

\end{enumerate}