\subsection{Judo}

\begin{enumerate}

\item Essence of Judo
    \begin{itemize}
    \item Paralympic judo, a martial art adapted for athletes with visual impairments, is a dynamic and strategic sport that emphasizes throws, grappling, and groundwork techniques. 
    \item It demands strength, agility, balance, and precise timing, as athletes engage in close combat to try to throw or immobilize their opponents. 
    \item The sport showcases the incredible skill and determination of visually impaired athletes who have mastered the art of judo through dedicated training and adaptation.
    \end{itemize}

\item Rules, Equipment, and Competition
    \begin{itemize}
    \item Paralympic judo follows the core principles and techniques of traditional judo, with modifications to accommodate athletes with visual impairments. 
    \item The main difference is that contestants start the match gripping each other's judogi (uniform), and verbal cues are used to guide the athletes during the contest. 
    \item The competition format involves different weight categories, and matches are won by throwing the opponent to the ground with control, immobilizing them, or forcing them to submit.
    \end{itemize}

\item Categories and Classifications
    \begin{itemize}
    \item Paralympic judo is open to athletes with visual impairments, categorized into different classes based on their level of visual acuity. 
    \item The classifications range from B1 for athletes with no light perception to B3 for those with some residual vision. 
    \item This system ensures fair competition and allows visually impaired athletes to showcase their judo skills and compete against others with similar levels of impairment. 
    \end{itemize}

\end{enumerate}