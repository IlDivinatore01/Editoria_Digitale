\subsection{Shooting Para Sport}

\begin{enumerate}

\item Essence of Shooting Para Sport
    \begin{itemize}
    \item Shooting Para Sport is a test of precision, focus, and control, where athletes with physical impairments compete in various shooting disciplines using rifles, pistols, or shotguns. 
    \item It demands exceptional hand-eye coordination, steady nerves, and the ability to maintain composure under pressure. 
    \item The sport showcases the remarkable accuracy and skill of athletes with disabilities, proving that limitations can be overcome with dedication and training.
    \end{itemize}

\item Rules, Equipment, and Competition
    \begin{itemize}
    \item Shooting Para Sport follows the core rules of shooting, with adaptations made to accommodate athletes with impairments. 
    \item Athletes compete in different events based on their classification and the type of firearm used. 
    \item The competition format involves shooting at targets from specific distances, and scores are based on accuracy and precision. 
    \item Adaptive equipment, such as shooting stands and specialized grips, is used to enable athletes with various impairments to participate and compete.
    \end{itemize}

\item Categories and Classifications
    \begin{itemize}
    \item Shooting Para Sport employs a classification system that groups athletes based on the impact of their impairment on their shooting ability. 
    \item There are three main classes: 
        \begin{itemize}
        \item SH1 for athletes who can support the weight of the firearm with their arms
        \item SH2 for those who require a shooting stand for support
        \item SH3 for visually impaired athletes who use acoustic signals to aim
        \end{itemize}
    \item This classification system ensures fairness and allows athletes with different impairments to compete on a level playing field.
    \end{itemize}

\end{enumerate}