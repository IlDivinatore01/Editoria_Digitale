\subsection{Wheelchair Rugby}

\begin{enumerate}

\item Essence of Wheelchair Rugby
    \begin{itemize}
    \item Wheelchair rugby, also known as "murderball," is a fast-paced and full-contact team sport that showcases the power, agility, and strategic brilliance of athletes with impairments affecting their limbs and trunk. 
    \item Played on a basketball court, athletes use specially designed wheelchairs to carry, pass, and dribble a volleyball across the opponent's goal line. 
    \item The sport is a thrilling spectacle of collisions, athleticism, and teamwork, demonstrating the resilience and competitive spirit of athletes with disabilities.
    \end{itemize}

\item Rules, Equipment, and Competition
    \begin{itemize}
    \item Wheelchair rugby combines elements of basketball, rugby, and handball, with modifications for wheelchair users. 
    \item Each team consists of four players, and the game is played in four 8-minute quarters. 
    \item Players use their wheelchairs to block and tackle opponents, creating a physically demanding and strategic game. 
    \item The sport utilizes specialized wheelchairs designed for durability and maneuverability, and athletes wear protective gear to minimize the risk of injury.
    \end{itemize}

\item Categories and Classifications
    \begin{itemize}
    \item Wheelchair rugby employs a unique classification system that assigns point values to athletes based on their functional abilities. 
    \item The system considers factors such as trunk control, arm and hand function, and overall mobility. 
    \item Each team is allowed a maximum number of points on the court at any given time, ensuring a balance of abilities and promoting fair competition among teams.
    \end{itemize}

\end{enumerate}