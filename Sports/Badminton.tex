\subsection{Badminton}

\begin{enumerate}

\item Essence of Badminton
    \begin{itemize}
    \item Paralympic badminton is a fast-paced and dynamic sport that showcases the agility, reflexes, and tactical acumen of athletes with various impairments.
    \item It involves hitting a shuttlecock over a net using a racket, demanding quick movements, precise shots, and strategic gameplay. 
    \item The sport offers both singles and doubles competitions, providing opportunities for athletes to demonstrate their individual skills and teamwork.
    \end{itemize}

\item Rules, Equipment, and Competition
    \begin{itemize}
    \item Paralympic badminton follows the core rules of badminton, with modifications made to accommodate different impairments. 
    \item Athletes can compete in standing or wheelchair categories, and the court dimensions and net height might be adjusted for certain classifications. 
    \item The competition format typically includes group stages followed by knockout rounds, culminating in exciting finals where athletes battle for medals and glory.
    \end{itemize}

\item Categories and Classifications
    \begin{itemize}
    \item Paralympic badminton employs a classification system based on the athlete's physical impairment, ensuring fair competition among those with similar functional abilities. 
    \item The system uses a combination of letters and numbers to categorize athletes, such as WH for wheelchair users and SL for standing lower limb impairments. 
    \item The classifications consider factors like muscle power, range of movement, and balance, allowing athletes to compete against others with comparable levels of impairment and showcase their badminton skills.
    \end{itemize}

\end{enumerate}